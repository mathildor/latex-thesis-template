\documentclass[10pt, b5paper, twoside]{book}
\usepackage[inner=1.8cm, outer=2.5cm, bottom=3.0cm]{geometry} % Spacing on each page - since two page style there are more space on outer border than inner boarder

\usepackage[parfill]{parskip}  % To begin paragraphs with an empty line rather than an indent
\usepackage{float}%To allow [H] after figures/tables ++ to force placement
\usepackage{lmodern}
\usepackage[utf8]{inputenc} %Allow norwegian letters
\usepackage[T1]{fontenc} % norsk tegnsett (æøå)
\usepackage[english]{babel}

%\usepackage{hyperref}
\usepackage[pdftex]{hyperref}

\usepackage{amsmath} %Math
\usepackage{amsthm} %Math?
\usepackage{textcomp}%Get math symbols like circled R

\usepackage{emptypage} %Removes pagenumber on empty pages 
\usepackage{setspace} %To use doublespacing / singlespacing ++

\usepackage[nonumberlist,toc]{glossaries}%Adding glossary, and adding it to contents. "nonumberlist" removes page reference in glossary
\usepackage[toc, page]{appendix} %Adding appendix

%Bibliography definitons
\usepackage{natbib}
\usepackage{apalike}
\usepackage{url} %Get url in referencelist


\usepackage[table,xcdraw]{xcolor} %Adding color to tabels

%AA------------------------------- Figure layout --------------------------------
\usepackage{graphicx} % To include images
\graphicspath{ {figures/} } %To add list of figures
\usepackage[font=footnotesize, labelfont=bf]{caption} %Setting size of figure text, and making figure numbering bold
\captionsetup{width=0.9\linewidth} %Setting global width of figure text (can also be set specifically for one figure)
\usepackage{caption}
\usepackage{subcaption} %To add caption for each independend side-by-side figure

%AA ---------------------------- CODE LAYOUT ------------------------------
% Inserting code and making it look readable (adding numbering, colours and formatting)
\usepackage{listings} %insert code style
\usepackage{color}
\definecolor{codegreen}{rgb}{0,0.6,0}
\definecolor{codegray}{rgb}{0.5,0.5,0.5}
\definecolor{codepurple}{rgb}{0.58,0,0.82}
\definecolor{backcolour}{rgb}{0.95,0.95,0.92}
\lstdefinestyle{pythonstyle}{
	backgroundcolor=\color{backcolour},   
	commentstyle=\color{codegreen},
	keywordstyle=\color{blue},
	numberstyle=\tiny\color{codegray},
	stringstyle=\color{codepurple},
	basicstyle=\footnotesize,
	breakatwhitespace=false,         
	breaklines=true,                 
	captionpos=b,                    
	keepspaces=true,                 
	numbers=left,                    
	numbersep=5pt,                  
	showspaces=false,   %Makes underscore look correct             
	showstringspaces=false,
	showtabs=false,                  
	tabsize=2
}
%creating new language with everything defined in python, but adding more keywords
\lstdefinelanguage{Python-tf}{
	language = {Python},
	morekeywords = {with, as},
}

%Making space between background box and content
\usepackage{fancyvrb}
\usepackage{framed}
\lstset{
	style=pythonstyle,
	framexleftmargin=1pt,
	framextopmargin=8pt,
	framexbottommargin=6pt, 
	frame=tb, framerule=0pt,
	%Setting arrow after forced line break in code
	postbreak=\raisebox{0ex}[0ex][0ex]{\ensuremath{\color{red}\hookrightarrow\space}}
}

%Get backround color on inline listing (code)
\usepackage{xpatch}
\usepackage{realboxes}
\makeatletter
\xpretocmd\lstinline{\Colorbox{backcolour}\bgroup\appto\lst@DeInit{\egroup}}{}{}
\makeatother


%AA-----------------------------  Header and footer layout--------------------------
\usepackage{fancyhdr}
\pagestyle{fancy}
\fancyhf{} % Clear all header and footer fields

%L = left, R = right, C=center, O = odd page, E = even page
\fancyhead[LE]{\leftmark} %\leftmark sets number and name of chapter
\fancyfoot[LE,RO]{\thepage} %\thepage gives page number

% redefinition of the plain style: Otherwise first page of new chapter gets original plain style
\fancypagestyle{plain}{%
	\fancyhf{} % clear all header and footer fields
	\renewcommand{\headrulewidth}{0pt} %Remove header line on new 
	\fancyfoot[LE,RO]{\thepage} %\ "thepage" gives page number
}

%AA-------------------- TOC layout ----------------------------------------------------
\usepackage{titletoc}

\titlecontents{chapter}[1.5pc]
{\filright}
{\contentslabel{1.5pc}}{\hspace*{-1.5pc}}
{\titlerule*[0.7pc]{.}\contentspage}

\titlecontents{section}[3.5pc]
{\addvspace{-0.4pc}\filright\footnotesize} %space between lines
{\contentslabel{2pc}}{\hspace*{-2pc}}
{\titlerule*[0.7pc]{.}\contentspage} %space between dots

\titlecontents{subsection}[4.5pc]
{\addvspace{-0.4pc}\filright\footnotesize} %space between lines
{\contentslabel{2pc}}{\hspace*{-2pc}}
{\titlerule*[0.7pc]{.}\contentspage} %space between dots

%Set levels in contents:
\setcounter{secnumdepth}{4}
\setcounter{tocdepth}{2}



%AA-------------------- Chapter header layout ------------------------------------
\usepackage{titlesec}
%\titleformat{\chapter}[hang] 

\usepackage{titlesec, blindtext, color}
\definecolor{gray75}{gray}{0.75}
\newcommand{\hsp}{\hspace{20pt}}

%Setting layot to line between number and name, setting the color to gray and adding space
\titleformat{\chapter}[hang]{\Huge\bfseries}{\thechapter\hsp\textcolor{gray75}{|}\hsp}{0pt}{\Huge\bfseries}

%{\normalfont\huge\bfseries}{\chaptertitlename\ \thechapter. }{0.7em}{}%[{\titlerule[0.2pt]}] %underline

\makeatletter %Decide layout for chapter header
\renewcommand{\@chapapp}{}% Remove "Chapter" before chap number
\newenvironment{chapquote}[2][2em]
%------------------------------------------------------------------


