\chapter{Recommendations}

\section{Latex editor}
I would advice running latex locally, instead of through shareLatex. It makes it easier to use to keep track of references. To easy keep backups of the work, I create a github repository. 

A latex editor that I have good experiences with is TexStudio: http://www.texstudio.org/. It gives good error-messages, and you can easily set hotkeys for inserting graphics, tables and so on.

\section{Reference manager}

To keep track of all the references used, a reference manager should be used. Mendeley is free and works great! 
It automatically creates a bib-file that you can add to your project in the "rapport.tex" file.

\section{Bibliography}
Apalike is a good style for bibliographies, and should be used. However, there is one problem with the style when referencing web-pages. It is on old style, and it therefore does not support adding the field "Date accessed" or URL. A hack for this is adding the URL to the field "Medium", and at the end of the URL add "Date Accessed: **-**-20**". The URL wont work as a link anymore, but it doesn't really matter since the thesis are printed anyways.